\documentclass[11pt]{article}
%\usepackage{graphicx} %Loading the package

\usepackage{biblatex}
\addbibresource{references.bib}




%\title{The first open access and \%100 reproducible PhD thesis} 
%\title{Pushing forward open access and reproducible science} 
%Wed 22 Jan 17:32:27 GMT 2020
%\title{The first open access and \%100 reproducible PhD thesis
%but this is only the begining} 
%Wed 22 Jan 18:18:17 GMT 2020
%\title{The journey of a PhD thesis and its challenges 
%to make it open accessible and reproducible} 
%Fri 24 Jan 08:07:26 GMT 2020

%\title{
%The first open access and 100\% reproducible 
%PhD thesis}
%%Sat 25 Jan 17:22:17 GMT 2020

%\title{
%The journey of the first open access and 100\% reproducible 
%PhD thesis}
%Tue 28 Jan 09:11:11 GMT 2020

%\title{
%Lessons from the first open access and 100\% reproducible 
%PhD thesis}
%%Tue 28 Jan 09:12:17 GMT 2020

\title{
Lessons from the 1st open accessible and 100\% reproducible 
PhD thesis}
%Sun  9 Feb 18:49:57 GMT 2020

\author{Miguel Xochicale}
\date{\today}

\begin{document}
\maketitle

%Surprisingly, %and indepently of the area of research, 
%last July 2019, no one to my knowledge have ever published 
%an open access and reproducible thesis 
%since the establishment of University of Birmingham in 1900. 
%%and with the confirmation of the library services of UoB
To my knowledge, and with the confirmation of the library services,
no one has ever published an open accessible and reproducible thesis 
since the establishment of University of Birmingham in 1900.  
That said, in this post, I would like to share the journey 
for such PhD thesis
%to
%see its light after the examiners accepted the major corrections 
%on June 2019 and
that along with the main limitations of time and funding, 
I also faced some other challenges %, for instance, %when 
such as being a minority and a non-native English speaker.  
%having English d as a native English speaker.
%and limitation such as time, 
%budget-wise, non-native English speaker, and being a minority. 
Those were the variables that accompany me in such a journey 
but 
always with main goal of making the science that 
I always dreamed of, science which is 
%keep me pushing forward to put together such thesis 
%that tackle basic issues of doing science 
reproduciblity and open accessesible to anyone anywhere.

%My hope is that with this post, my PhD thesis has more 
%visibility and perhaps others can see the do and dont's,
%barriers of publishing this sort of thesis.
%can be used a model for other research in the UK and
%worthwhile.

%Additionally to the little contributions to new knowledge, 
%the effort to make a good narravite in the story of my PhD thesis, 
%one also have to think of making better science. 
%How to do better science with various limitations 
%(i.e. time, budget-wise, non-native English speaker, etc)?

%my case also 
%Putting together a PhD thesis can be daunting but 
%additionally to that, to make better science, one 
%have to pushed a bit more to make it open accessible and reproducible.
%That is my journey, a minority, limited resources, etc.

%when I embark myself on a new jouney called PhD 
%that naively make me think on executing/doing the most marevelous
%research, 
%which as a naively I wanted to research marvelous ideas, 
%performing cool experiments, write up expectacular manuscripts and 
%alos present those results in prestige consferece.
%urns out to be that starting from scratch, those previous points 
%are still far in my lifetime but I have made a little bit of progress


%%%%%%%%%%%%%%%
%FIRST YEAR
I started a four-year journey of my PhD on November 2014 
by, firstly, struggling to put together sentences in a language
different from my mother tongue, then using new programming 
languages along with many open access software and tools to choose 
such as GNU-Linux distributions, version control tools, etc. 
%%%%%%%%%%%%%%%
%SECOND YEAR
Then in May 2015, the second year of my PhD, I joined twitter and
with the desired of looking for like-minded humans. So in that
exploration, I started to follow 
Olivia Guest (https://twitter.com/o\_guest) and all her nice Github contributions, 
then I found the repositories of 
Severin Lemaignan, a robotics who is considering himself
an open-source enthusiast (https://github.com/severin-lemaignan). 
By then, I made the decision to use R instead of python because 
the advances of reading huge datasets. Also that year I met 
Jon Tennant (https://twitter.com/protohedgehog) 
who is regarding himself as the batman of open science
and who was an eye-opener on the status of making open 
science.
% and resonating with like-minded people.
%%%%%%%%%%%%%%%
%THIRD YEAR
Then the third year of my PhD was mainly about refining experiments, 
collecting, analysing data and finding ways to make them open accessible 
and reproducible but not least important than thinking seriously 
of writing up my thesis. 
%%%%%%%%%%%%%%%
%FOUR YEAR
The fourth year then come along with the challenge of 
putting together a thesis where 
all the previous open-source enthusiast, reference for code and websites 
made a great influence in the final version of my thesis such as 
the discovery of the repository starry 
by Rodrigo Luger (https://github.com/rodluger/starry)
with the beautiful embedded links to the python code
and also various Github websites and repositories the NVIDIA toronto AI lab
(https://github.com/nv-tlabs/meta-sim).



%with videos and zenodo links.
%PLOT for 
%* number of thesis with code 
%* number of thesis as open access
%* number of thesis that are reproducible

All those four years allow me to get into open accessible science
and to follow and learn from other in order to put together 
repositories for the \LaTeX project, data and code, website, zenodo links 
and video in youtube for the public dissemination of the first
open accessible and \%100 reproducible since the establishment of 
University of Birmingham in 1900.
 
% THE FUTURE
But this is not the end of this work, as now I am wondering 
where is the future of open access and reproducible science.
I guess in the closer future, there is something called actions 
which can be used in
%well, recently I have discovere tools called actions either 
in azure, Gitlab or Github.
% which allow you to do 
%continuous development and continuous integration.
%With that in mind, 
But my only hope with this is that others also consider 
to change the direction of how science has been done by 
using more open access software and make science better 
by making it
open accessible and \%100 reproducible 
%PhD thesis since the establishment of University of Birmingham 
\cite{xochicale2019-github}.


%Fri  8 Nov 17:23:25 GMT 2019
%Growing up in mexico loving being an experimentalist

\printbibliography

\end{document}
