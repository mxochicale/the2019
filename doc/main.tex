\documentclass[11pt]{article}
%\usepackage{graphicx} %Loading the package

\usepackage{biblatex}
\addbibresource{references.bib}
\usepackage{datetime2}

%\title{The first open access and \%100 reproducible PhD thesis} 
%\title{Pushing forward open access and reproducible science} 
%Wed 22 Jan 17:32:27 GMT 2020
%\title{The first open access and \%100 reproducible PhD thesis
%but this is only the begining} 
%Wed 22 Jan 18:18:17 GMT 2020
%\title{The journey of a PhD thesis and its challenges 
%to make it open accessible and reproducible} 
%Fri 24 Jan 08:07:26 GMT 2020

%\title{
%The first open access and 100\% reproducible 
%PhD thesis}
%%Sat 25 Jan 17:22:17 GMT 2020

%\title{
%The journey of the first open access and 100\% reproducible 
%PhD thesis}
%Tue 28 Jan 09:11:11 GMT 2020

%\title{
%Lessons from the first open access and 100\% reproducible 
%PhD thesis}
%%Tue 28 Jan 09:12:17 GMT 2020

%\title{
%Lessons from the 1st open accessible and 100\% reproducible
%PhD thesis}
%%Sun  9 Feb 18:49:57 GMT 2020

%\title{
%The first completely open engineering-based PhD thesis.}
%%Tue 16 Jun 00:25:06 BST 2020

%\title{
%The first full open engineering-based PhD thesis.}
%%Mon  6 Jul 15:42:23 BST 2020

%\title{
%Open engineering-based PhD thesis}
%%Mon  6 Jul 21:43:05 BST 2020

\title{
Fully Open Engineering-based PhD thesis}
%Wed  8 Jul 22:50:42 BST 2020




\author{Miguel Xochicale}
\date{\DTMNow}

\begin{document}
\maketitle

%%Surprisingly, %and indepently of the area of research,
%%last July 2019, no one to my knowledge have ever published
%%an open access and reproducible thesis
%%since the establishment of University of Birmingham in 1900.
%%%and with the confirmation of the library services of UoB
%To my knowledge, and with the confirmation of the library services,
%no one has ever published an open accessible and reproducible thesis
%since the establishment of University of Birmingham in 1900.
%That said, in this entry, I would like to share the journey
%for such PhD thesis
%%to
%%see its light after the examiners accepted the major corrections
%%on June 2019 and
%that along with the time-wise and budget-wise limitations,
%I also faced some other challenges %, for instance, %when
%such as being a minority and a non-native English speaker.
%%having English d as a native English speaker.
%%and limitation such as time,
%%budget-wise, non-native English speaker, and being a minority.
%Those were the variables that accompany me in such a journey
%but
%always with main goal of making the science that
%I always dreamed of, science which is
%%keep me pushing forward to put together such thesis
%%that tackle basic issues of doing science
%reproduciblity and open accessesible to anyone anywhere.

%My hope is that with this post, my PhD thesis has more 
%visibility and perhaps others can see the do and dont's,
%barriers of publishing this sort of thesis.
%can be used a model for other research in the UK and
%worthwhile.

%Additionally to the little contributions to new knowledge, 
%the effort to make a good narravite in the story of my PhD thesis, 
%one also have to think of making better science. 
%How to do better science with various limitations 
%(i.e. time, budget-wise, non-native English speaker, etc)?

%my case also 
%Putting together a PhD thesis can be daunting but 
%additionally to that, to make better science, one 
%have to pushed a bit more to make it open accessible and reproducible.
%That is my journey, a minority, limited resources, etc.

%when I embark myself on a new jouney called PhD 
%that naively make me think on executing/doing the most marevelous
%research, 
%which as a naively I wanted to research marvelous ideas, 
%performing cool experiments, write up expectacular manuscripts and 
%alos present those results in prestige consferece.
%urns out to be that starting from scratch, those previous points 
%are still far in my lifetime but I have made a little bit of progress

What is the kind of science you always dream of? Is it not one that is
along the lines of reproducibility, inclusiveness, transparency,
reusability  and open accessibility?
%In this post, I would like to share experiences
%and the people who inspire me 
%to give answers 
%to the above questions in the context of 
In trying to answer such questions, I am sharing in this post
my four-year PhD journey that ended up in my thesis which, 
to my knowledge and with the confirmation of the library services,
is the first open accessible and reproducible thesis
since the establishment of University of Birmingham in 1900.

%As those who have been, or might consider to going, thought a PhD, 
%I can say that the journey in itself is not a bed of roses 
%and you might encounter some additional limitations such as 
%time-wise and budget-wise.
%%%%%%%%%%%%%%%
%FIRST YEAR
So, even knowing the time-wise and budget-wise limitations,
I adventured myself to start my PhD in November 2014, 
by, firstly, struggling to put together sentences in a language
different from my mother tongue, Spanish, to then 
the use of new programming languages (e.g. R, python, julia, etc.), 
open access software applications and tools 
(e.g., GNU-Linux distributions, version control tools, etc.).  
%%%%%%%%%%%%%%%
%SECOND YEAR
Then in May 2015, when I was in the second year of my PhD, I joined twitter 
with the desired of handing-out with like-minded human beings. 
In such stage of exploration/procrastination, I bumped into the amazing 
profile of Olivia Guest (https://twitter.com/o\_guest) and
all her nice contributions in Github. 
Then, by the end of 2015, I founded the repositories of 
Severin Lemaignan, a robotics who is considering himself
an open-source enthusiast (https://github.com/severin-lemaignan),
who also was as well a role model on how to write up beamer
presentation, ros packages, etc. 
%By then, I made the decision to use R programming language 
%instead of python because of the computing performance with huge datasets. 
Also that year I met Jon Tennant (https://twitter.com/protohedgehog) 
%who sadly passed recently away but leaving a great legacy 
%for those who believe that knowledge is a human right. 
%Through his tweets make myself and others an  
%who is regarding himself as the batman of open science and was 
who was an eye-opener on making knowledge is a human right
and discovering other people like himself with the endeavour of making open science.
% and resonating with like-minded people.
%%%%%%%%%%%%%%%
%THIRD YEAR
Then by the third year of my PhD, I was mainly refining experiments, 
collecting, analysing data and finding ways to make them open 
accessible and reproducible with the use of GitHub.
%but not least important than thinking seriously of writing up my thesis. 
%%%%%%%%%%%%%%%
%FOUR YEAR
Hence, I reached the scary fourth year that come along with the 
challenge of putting together a thesis combining all the previous 
experiences and skills in open science to the be able to 
%and the following great influences were 
merge them into the final version of my thesis.
And the cherry on the cake was the 
%such as open-source enthusiast, references for code, s well as 
the discovery of the repository starry 
by Rodrigo Luger (https://github.com/rodluger/starry)
that beautifully embedded links to the python code
and various Github websites and repositories such 
the NVIDIA toronto AI lab (https://github.com/nv-tlabs/meta-sim)
that inspire me to the final version of my thesis.
%with videos and zenodo links.
%PLOT for 
%* number of thesis with code 
%* number of thesis as open access
%* number of thesis that are reproducible
% THE FUTURE
So, that was my four-years PhD journey along with the 
accumulated experiences that allow me to get into open accessible,
reproducible and reproducible science
%and to follow and learn from other in order to 
by putting together GitHub repositories for the \LaTeX project, 
data and code, website, zenodo links as well as a video in youtube
to just 
%But this is not the end of this work, as now 
%I am currently wondering 
%where is
realise that it is only the beginning to keep pushing forward 
the future of open access and reproducible science.
%A closer future where thesis can be openly reviewed with the use of 
%%I guess in the closer future, there is something called 
%actions in in
%%well, recently I have discovere tools called actions either 
%in azure, Gitlab or Github.
%% which allow you to do 
%%continuous development and continuous integration.
%%With that in mind, 

Recently, Heise and Pearce 2020 \cite{heise2020} pointed out that 
in the context of open access thesis there 
are still various challenges 
in the existing system of formal scientific communication 
(e.g. performance evaluation of scientific work, speed in the communication process,
efficiency, defect resistance and quality assurance, dissemination and accessibility, 
quality and prevention of misuse and scientific misconduct)
and some has been tackled on above entry. 
So my hope is that others can be inspired and can consider 
to change the direction of how science has been done by  
pursing a similar approach to get us closer to a science that is
along the lines of 
reproducibility, inclusiveness, transparency,
reusability  and open accessibility \cite{xochicale2019-github}.

%more open access software 
%and make science better by making it
%open accessible and \%100 reproducible 
%PhD thesis since the establishment of University of Birmingham 
%\cite{heise2020}, 
%\cite{xochicale2019-github}.
%And you are not along anymore
%as there is a nice community of Early Career Researchers 
%such RIOTs club or ReproducibliTEE
% and twitter \#OpenScience is always willing to help and every second 
%new things are happening. 

\printbibliography

\end{document}
