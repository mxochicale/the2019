\documentclass[11pt]{article}
\usepackage[a4paper, total={6in, 10in}]{geometry}
\usepackage{natbib}
%\usepackage{graphicx} %Loading the package
%\graphicspath{{../src/sinewaves/}}

\usepackage{xspace}
\newcommand{\R}{\textsf{R}\xspace}

%% Codelinks
\usepackage{hyperref}
\usepackage{fontawesome}
\usepackage{color}
\definecolor{linkcolor}{rgb}{0.1216,0.4667,0.7059} % blue for ONLINE version
\definecolor{linkbw}{rgb}{0.0, 0.0, 0.0} % black for PRINTER version
\newcommand{\codeicon}{{\color{linkcolor}\faFileCodeO}}
\newcommand{\codecolorlink}[1]{\href{#1}{\codeicon}}


%\title{The first open access and \%100 reproducible PhD thesis} 
%\title{Pushing forward open access and reproducible science} 
%Wed 22 Jan 17:32:27 GMT 2020
%\title{The first open access and \%100 reproducible PhD thesis
%but this is only the begining} 
%Wed 22 Jan 18:18:17 GMT 2020
%\title{The journey of a PhD thesis and its challenges 
%to make it open accessible and reproducible} 
%Fri 24 Jan 08:07:26 GMT 2020

%\title{
%The first open access and 100\% reproducible 
%PhD thesis}
%%Sat 25 Jan 17:22:17 GMT 2020

%\title{
%The journey of the first open access and 100\% reproducible 
%PhD thesis}
%Tue 28 Jan 09:11:11 GMT 2020

\title{
Lessons from the first open access and 100\% reproducible 
PhD thesis}
%Tue 28 Jan 09:12:17 GMT 2020


\author{Miguel Xochicale}
\date{\today}

\begin{document}
\maketitle

Surprisingly, %and indepently of the area of research, 
last July 2019, no one to my knowledge have ever published 
an open access and reproducible thesis 
since the establishment of University of Birmingham in 1900. 
%and with the confirmation of the library services of UoB 
That said, in this post I will share my journey for such PhD thesis
%to
%see its light after the examiners accepted the major corrections 
%on June 2019 and
that along with the limitations of time and funding, 
I also faced some other challenges, for instance, %when 
being a minority and non-native English speaker.  
%having English d as a native English speaker.
%and limitation such as time, 
%budget-wise, non-native English speaker, and being a minority. 
Those were the variables that acompanimy me in this journey 
but always with main one of making the science that 
I always dreamed of, the science which is 
%keep me pushing forward to put together such thesis 
%that tackle basic issues of doing science 
reproduciblity and open accessesible to anyone anywhere.

%My hope is that with this post, my PhD thesis has more 
%visibility and perhaps others can see the do and dont's,
%barriers of publishing this sort of thesis.
%can be used a model for other research in the UK and
%worthwhile.

%Additionally to the little contributions to new knowledge, 
%the effort to make a good narravite in the story of my PhD thesis, 
%one also have to think of making better science. 
%How to do better science with various limitations 
%(i.e. time, budget-wise, non-native English speaker, etc)?

%my case also 
%Putting together a PhD thesis can be daunting but 
%additionally to that, to make better science, one 
%have to pushed a bit more to make it open accessible and reproducible.
%That is my journey, a minority, limited resources, etc.

%when I embark myself on a new jouney called PhD 
%that naively make me think on executing/doing the most marevelous
%research, 
%which as a naively I wanted to research marvelous ideas, 
%performing cool experiments, write up expectacular manuscripts and 
%alos present those results in prestige consferece.
%urns out to be that starting from scratch, those previous points 
%are still far in my lifetime but I have made a little bit of progress


%FIRST YEAR
So, I started the four-year journey of my PhD on November 2014 
by, firstly, struggling to put together sentences in a language
different from my mother tongue, then using new programming 
languags along with many open access software and tools such as 
Ubuntu, GitHub, python, julia, R, octave, etc. 
%SECOND YEAR
Then for the second year, I joined twitter in May 2015, and 
I as so thrilled to start following Olivia Guest 
(https://twitter.com/o\_guest) her Github contributions, 
then I found the repositories of 
Severin Lemaignan, a robotics who is considering himself
an open-source enthusiast (https://github.com/severin-lemaignan),
then decided to user R because of the advances of reading huge datasets
than python. Also that year I met the Jon Tennant 
(https://twitter.com/protohedgehog) who was an open-eye by reading 
his twitter post and resonating with like-minded people.
%THIRD YEAR
Then the third year of my PhD was mainly about refining experiments, 
collecting and its analysis data and also making it open accessible 
and reproducible and start thinking seriously of writing up
my thesis. 
%FOUR YEAR
The fourth year then come along with the challenge of 
putting together a thesis where I discovered the repo of 
starry by Rodrigo Luger (https://github.com/rodluger/starry)
and the beautiful embedded links to the python code
and various Github websites 
of the repos of the NVIDIA toronto AI lab
(https://github.com/nv-tlabs/meta-sim).
%with videos and zenodo links.
%PLOT for 
%* number of thesis with code 
%* number of thesis as open access
%* number of thesis that are reproducible


% THE FUTURE
Then to close this post, I am wondering where is the future of 
open access and reproducible science, I guess a closer 
future is in something called actions that can be used in
%well, recently I have discovere tools called actions either 
in azure, gitlab or github which allow you to do 
continuous development and continuous integration.
With that in mind, my hope is that others also try to have a go to
open access software and make science better by making it
open accessible and \%100 reproducible 
%PhD thesis since the establishment of University of Birmingham 
\citep{xochicale2019-github}.




%Fri  8 Nov 17:23:25 GMT 2019
%Growing up in mexico loving being an experimentalist


\bibliography{references}
\bibliographystyle{apalike}

\end{document}
